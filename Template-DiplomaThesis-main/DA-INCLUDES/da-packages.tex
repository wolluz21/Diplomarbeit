% Deutsche Silbentrennung
\usepackage[ngerman]{babel}
% Deutsche Umlaute
\usepackage[utf8]{inputenc}
\usepackage[T1]{fontenc}
% Höhe der Kopf- und Fußzeile definieren
\usepackage{geometry}
\geometry{verbose,a4paper,tmargin=25mm,bmargin=25mm,lmargin=25mm,rmargin=25mm, headsep=0.5cm}
% Weitere Packages
\usepackage[titletoc]{appendix}
\usepackage{csquotes}
\usepackage{blindtext}
\usepackage{setspace} 
\usepackage[table]{xcolor}
\usepackage{xcolor}
\usepackage{graphicx}
\usepackage{tikz}
\usepackage{ifthen}
\usepackage[]{ragged2e}
\usepackage{lmodern}
\usepackage{fancyhdr}
\usepackage{xcolor,colortbl}
\usepackage{hyperref}
\usepackage{titlesec}
\setlength\parindent{0pt}
\usepackage[font={footnotesize}]{caption}
% imports für tabellen
\usepackage{longtable}
\usepackage{multirow}

% SI-Einheiten
\usepackage{siunitx}
\sisetup{locale = DE,per-mode=fraction}

% Literaturvereichnis / Zitate
\usepackage[
    backend=biber,
    style=apa,     %apa or numeric
    autocite=inline]{biblatex}
    
% deutsche Anfuehrungszeichen 
\newcommand{\anf}[1]{\glqq #1\grqq{}} %SH_220105 eingefügt

% Links im Inhaltsverzeichnis
\hypersetup{
    colorlinks,
    citecolor=black,
    filecolor=black,
    linkcolor=black,
    urlcolor=black
}
% Schriftart einbinden
\usepackage{sourcesanspro} % roman is default font
\usepackage{courier} % used for \code{arg1}  \inlinecode

% Inhaltsverzeichnis umbennen
\addto\captionsenglish{
  \renewcommand{\contentsname}
    {Inhaltsverzeichnis}
}
\addto\captionsngerman{
  \renewcommand{\contentsname}
    {Inhaltsverzeichnis}
}

% Literaturverzeichnis
%\usepackage[numbers]{natbib} 

% Caption-Abkürzungen
\addto\captionsenglish{\renewcommand{\figurename}{Abb.}}
\addto\captionsngerman{\renewcommand{\figurename}{Abb.}}

% Fußnote
\usepackage[bottom]{footmisc}

% Package für Source Code
\usepackage{listings}
\usepackage{listing}

% C-Sharp
\definecolor{Green}{rgb}{0, 0.3, 0}
\definecolor{DarkCyan}{rgb}{0, 0.545, 0.545}
\definecolor{Navy}{rgb}{0, 0, 0.5}
\definecolor{Teal}{rgb}{0, 0.5, 0.5}
\definecolor{DarkGray}{gray}{0.66}
\definecolor{Olive}{rgb}{0.5, 0.5, 0}
\definecolor{Pink}{rgb}{1.0, 0.75, 0.8}
\definecolor{DeepPink}{rgb}{1, 0.08, 0.58}
\definecolor{Brown}{rgb}{0.65, 0.165, 0.165}
\definecolor{DarkViolet}{rgb}{0.58, 0, 0.83}
\definecolor{SaddleBrown}{rgb}{0.55, 0.27, 0.07}
\definecolor{base0}{RGB}{131,148,150}
\definecolor{base01}{RGB}{88,110,117}
\definecolor{base2}{RGB}{238,232,213}
\definecolor{sgreen}{RGB}{133,153,0}
\definecolor{sblue}{RGB}{38,138,210}
\definecolor{scyan}{RGB}{42,161,151}
\definecolor{smagenta}{RGB}{211,54,130}
\newcommand\digitstyle{\color{smagenta}}
\newcommand\symbolstyle{\color{base01}}
\lstdefinestyle{solarizedcsharp} {
  language=[Sharp]C,
  frame=lr,
  linewidth=150mm,
  breaklines=true,
  tabsize=2,
  rulecolor=\color{base2},
  basicstyle=\footnotesize\ttfamily,
  commentstyle=\color{base01},
  morecomment=[s][\color{base01}]{/*+}{*/},
  morecomment=[s][\color{base01}]{/*-}{*/},
  morekeywords={  abstract, event, new, struct,
                as, explicit, null, switch,
                base, extern, object, this,
                bool, false, operator, throw,
                break, finally, out, true,
                byte, fixed, override, try,
                case, float, params, typeof,
                catch, for, private, uint,
                char, foreach, protected, ulong,
                checked, goto, public, unchecked,
                class, if, readonly, unsafe,
                const, implicit, ref, ushort,
                continue, in, return, using,
                decimal, int, sbyte, virtual,
                default, interface, sealed, volatile,
                delegate, internal, short, void,
                do, is, sizeof, while,
                double, lock, stackalloc,
                else, long, static,
                enum, namespace, string, var},
  keywordstyle=\color{sgreen},
  showstringspaces=false,
  stringstyle=\color{scyan},
  identifierstyle=\color{sblue},
  extendedchars=true,
}

\definecolor{bluekeywords}{rgb}{0,0,1}
\definecolor{greencomments}{rgb}{0,0.5,0}
\definecolor{redstrings}{rgb}{0.64,0.08,0.08}
\definecolor{xmlcomments}{rgb}{0.5,0.5,0.5}
\definecolor{types}{rgb}{0.17,0.57,0.68}

\usepackage{listings}
\lstdefinestyle{mycsharp} {
language=[Sharp]C,
captionpos=b,
showspaces=false,
showtabs=false,
breaklines=true,
showstringspaces=false,
breakatwhitespace=true,
backgroundcolor=\color{gray!10},
escapeinside={(*@}{@*)},
commentstyle=\color{greencomments},
morekeywords={  abstract, event, new, struct,
                as, explicit, null, switch,
                base, extern, object, this,
                bool, false, operator, throw,
                break, finally, out, true,
                byte, fixed, override, try,
                case, float, params, typeof,
                catch, for, private, uint,
                char, foreach, protected, ulong,
                checked, goto, public, unchecked,
                class, if, readonly, unsafe,
                const, implicit, ref, ushort,
                continue, in, return, using,
                decimal, int, sbyte, virtual,
                default, interface, sealed, volatile,
                delegate, internal, short, void,
                do, is, sizeof, while,
                double, lock, stackalloc,
                else, long, static,
                enum, namespace, string, var, 
                value, get, set, partial},
keywordstyle=\color{bluekeywords},
stringstyle=\color{redstrings},
basicstyle=\ttfamily\small,
identifierstyle=\color{sblue},
frame=single,
linewidth=159mm,
breaklines=true,
rulecolor=\color{black!40},
tabsize=4,
framexleftmargin=-0mm
}

% Python-Code
%\usepackage{pythonhighlight}

% C-Code
\definecolor{mGreen}{rgb}{0,0.6,0}
\definecolor{mGray}{rgb}{0.5,0.5,0.5}
\definecolor{mPurple}{rgb}{0.58,0,0.82}
\definecolor{backgroundColour}{rgb}{0.95,0.95,0.92}

\lstdefinestyle{myc}{
    backgroundcolor=\color{gray!10},
    commentstyle=\color{mGreen},
    keywordstyle=\color{magenta},
    numberstyle=\tiny\color{mGray},
    stringstyle=\color{mPurple},
    breakatwhitespace=true,         
    breaklines=true,                 
    captionpos=b,  
    basicstyle=\ttfamily\small,                  
    keepspaces=true,                 
    linewidth=159mm,
    frame=single,  
    breaklines=true,
    rulecolor=\color{black!40},
    tabsize=4,
    framexleftmargin=-0mm,      
    showspaces=false,                
    showstringspaces=false,
    showtabs=false,                  
    language=C
}

% XML-Code
\definecolor{dkgreen}{rgb}{0,0.6,0}
\definecolor{gray}{rgb}{0.5,0.5,0.5}
\definecolor{mauve}{rgb}{0.58,0,0.82}
\definecolor{gray}{rgb}{0.4,0.4,0.4}
\definecolor{darkblue}{rgb}{0.0,0.0,0.6}
\definecolor{lightblue}{rgb}{0.0,0.0,0.9}
\definecolor{cyan}{rgb}{0.0,0.6,0.6}
\definecolor{darkred}{rgb}{0.6,0.0,0.0}

\lstdefinestyle{myxml}
{
  morestring=[b],
  morestring=[s]{>}{<},
  morecomment=[s]{<?}{?>},
  basicstyle=\ttfamily\small,
  keepspaces=true,                 
  linewidth=159mm,
  frame=single,  
  breaklines=true,
  rulecolor=\color{black!40},
  backgroundcolor=\color{gray!10},
  tabsize=4,
  framexleftmargin=-0mm,      
  showspaces=false,                
  showstringspaces=false,
  showtabs=false,     
  stringstyle=\color{black},
  identifierstyle=\color{darkblue},
  keywordstyle=\color{cyan},
  captionpos=b,
  morekeywords={xmlns,xsi,noNamespaceSchemaLocation,type,id,x,y,source,target,version,tool,transRef,roleRef,objective,eventually}
}

\lstdefinestyle{mygeneral}
{
  morestring=[b],
  morestring=[s]{>}{<},
  morecomment=[s]{<?}{?>},
  basicstyle=\ttfamily\small,
  keepspaces=true,                 
  linewidth=159mm,
  frame=single,  
  breaklines=true,
  rulecolor=\color{white},
  backgroundcolor=\color{gray!10},
  tabsize=4,
  framexleftmargin=-0mm,      
  showspaces=false,                
  showstringspaces=false,
  showtabs=false,     
  captionpos=b,
}

% Umbennen von Code Listing
\renewcommand{\lstlistingname}{Quellcode}


% Bilder floaten
\usepackage{float}
\usepackage{wrapfig}
\newfloat{listing}{H}{loc}
\floatname{listing}{Listing}

% Zeilenumbruch in Tabelle
\usepackage{makecell}

% Blocksatz
\usepackage{microtype}

\setcounter{secnumdepth}{4}

\titleformat{\paragraph}
{\normalfont\normalsize\bfseries}{\theparagraph}{0.7em}{}
\titlespacing*{\paragraph}
{0pt}{3.25ex plus 1ex minus .2ex}{1.5ex plus .2ex}

\pagestyle{fancy}

% Schriftart festlegen
\renewcommand{\familydefault}{\rmdefault}  % Serifenschrift
%\renewcommand{\familydefault}{\sfdefault}  % Serifenlose Schrift

% Text in Kopfzeile für Vorwort festlegen
\renewcommand{\sectionmark}[1]{\markright{\thesection\ #1}}

% Fußzeilenlinie definieren
\renewcommand{\footrulewidth}{0.4pt}

% Autor-Kommando definieren
\newcommand{\TheAuthor}{}
\newcommand{\Author}[1]{\renewcommand{\TheAuthor}{#1}}

% Zellenhöhe für Tabelle
\renewcommand{\arraystretch}{1.3}

% Inline-Code
\newcommand{\inlinecode}{\texttt}
\let\oldtexttt\texttt
\newcommand{\code}[1]{\colorbox{lightgray!25}{\oldtexttt{#1}}} 
\renewcommand{\texttt}{\code}

\newenvironment{longcite}
{
 	\begin{quote}
 		\small
		\itshape
			}
			{	
	\end{quote}
}

\usepackage{ifthen}
\newboolean{isInformatik} % declaration
\newboolean{isEnglish} % declaration
\newboolean{isDoublePagePrinting} % declaration
\newboolean{isSansSerif} % declaration

\newcommand*{\inputLanguageText}[1]{
	\ifthenelse{\boolean{isEnglish}}{
		\input{DA-INCLUDES/EN/#1}
	}{
		\input{DA-INCLUDES/DE/#1}	
	}
}

\newcommand*{\inputDepartmentTitlePage}[1]{
	\ifthenelse{\boolean{isEnglish}}{
		%\renewcommand{\familydefault}{\sfdefault}
\begin{titlepage}
	\centering
	
	\begin{minipage}[t]{0.45\textwidth}
		\flushleft
  		\centering\raisebox{\dimexpr \topskip-\height}{
		\includegraphics[width=0.9\textwidth]{images/htl_logo.pdf}
		}
	\end{minipage}
	\begin{minipage}[t]{0.45\textwidth}
		\raggedright
  		\centering\raisebox{\dimexpr \topskip-\height}{
    		\includegraphics[width=0.7\textwidth]{images/inf_farbe_2.pdf} 
  		}
	\end{minipage}
	\vspace{1.5cm}	
	
	
	%\includegraphics[width=0.4\textwidth]{images/htl_logo.pdf}
	%\includegraphics[width=0.3\textwidth]{images/informatik_logo.png}\par\vspace{1.5cm}
	{\textbf {Höhere Technische Bundeslehranstalt Kaindorf an der Sulm}\par\vspace{0.3cm}}
	{\textbf {Abteilung Automatisierungstechnik}\par	\vspace{0.7cm}}
	\Large{\textbf{Diplomarbeit}}\par\vspace{0.2cm}
	\small im Rahmen der Reife- und Diplomprüfung\par
	\vspace{1.5cm}
	\Huge\textbf{\daTitle}\\
	\vspace{1cm}
	\includegraphics[width=0.4\textwidth]{images/inf_farbe_2.pdf}\\
	\vspace{1cm}
	\normalsize \daAuthorOne \par
	\normalsize \daAuthorTwo \par
	\normalsize \daAuthorThree \par
	\vspace{0,5cm}
	\daGrade\\ \daYear
	\vfill
	\begin{flushleft}
	\begin{tabbing}
	Diese Zeile wird \= noch \= gelöscht \=  \kill\vspace{0.2cm}Betreuer:\> \daSupervisorOne\par 
	\\\vspace{0.2cm}\> \daSupervisorTwo \par 
	\\\vspace{0.2cm}\> \daSupervisorThree \par 
	\\\vspace{0.2cm}Projektpartner:\> \daPartner \par
	\\Datum:\> \daDocDate
	\end{tabbing}
	\end{flushleft}
\end{titlepage} }{
		\input{#1/da-1.0_titlePage-de.tex}	
	}
}

\newcommand*{\doubleOrSinglePaged}{
	\ifthenelse{\boolean{isDoublePagePrinting}}{
		\fancyhead[L]{\ifthenelse{\isodd{\value{page}}}{}{\rightmark}}
		\fancyhead[R]{\ifthenelse{\isodd{\value{page}}}{\rightmark}{}}
		\cfoot{}
		\fancyfoot[L]{\ifthenelse{\isodd{\value{page}}}{\TheAuthor}{\thepage}}
		\fancyfoot[R]{\ifthenelse{\isodd{\value{page}}}{\thepage}{\TheAuthor}}
	}{
		% ********* begin einseitiger Druck *********** SH_220305
		\fancyhead[R]{\rightmark}
		\cfoot{\daTitle}
		\fancyfoot[R]{\thepage}
		\fancyfoot[L]{\TheAuthor}
	}
}

\newcommand*{\SerifOrSans}[1]{
	\ifthenelse{\boolean{isSansSerif}}{
		\renewcommand{\familydefault}{\sfdefault}
		\sffamily}{
		\renewcommand{\familydefault}{\rmdefault}
		\rmfamily
	}
}
